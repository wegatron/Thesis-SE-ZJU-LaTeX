%!TEX root = ../main.tex

\chapter{相关网格简化算法及其分析}
\section{QEM 网格简化算法}
QEM算法凭着其简单高效的特点是网格简化算法中主流算法之一。在QEM算法中,每一个模型被视为是由很多个离散的有限平面所组成,而模型上的每一个顶点就是其一环邻域(周围的一圈)平面的交点。因此,我们将顶点$v=[x,y,z,1]^T$和一系列平面关联起来,相当于每一个顶点$v$都应该在一个平面集合$\text{planes}(v)$的每个平面上。每一个$\text{plane}(v)$都可以表示为$p^Tv=0,p=[a,b,c,d]^T,a^2+b^2+c^2=0即ax+by+cz+d=0$,从而该顶点到这个平面$\text{plane}(v)$的距离的平方可表示为$(p^Tv)^2$即$v^Tpp^Tv$,每一个顶点到其平面集合的距离平方和为
\begin{equation}
  \Delta(v) = \sum_{p\in \text{planes}(v)}v^Tpp^T = v^TQ_vv
\end{equation}
我们称$\Delta(v)$为Quadric Error在初始的情况下$\forall v,\Delta(v)=0$。在QEM算法中,我们用这个Quadric
Error作为简化结果与原模型的距离衡量标准。当我每做一次顶点合并$(v_0, v_1) \to v̅$时,我们将这两个顶点的平面集合合并成为新的顶点的平面集合即planes(v̅) = planes(v 0 ) ∪ planes(v 1 ),从而会产生一个新的Quadric Error$\Delta(v̅)=v^T(Qv_0+Qv_1)v$。该算法以最小化Quadric Error的上界为目标,每次都是以贪心的策略从可合并的顶点对(构成一条边的两个顶点,或者两个距离小于一定阈值的顶点)中选取$\Delta(v̅)$最小的顶点对做合并。该算法实现起来非常简单,可以归纳为以下几步:
\begin{enumerate}
  \item 初始化每个三角网格上的顶点的矩阵Q;
  \item 通过$minimize v̅^T(Qv_0+Qv_1)v̅$,计算每一条边的两个顶点的最优合并点$v̅$ ,以$\Delta(v̅)$作为这条边合并的代价,以此一个最小优先级队列;
  \item 迭代地从最小优先级队列中取出一个顶点对合并,然后更新这个优先级队列。
\end{enumerate}
QEM方法不仅简单高效,而且就算在大程度的简化要求下,简化结果非常好地保持了原模型的视觉效果如图。
