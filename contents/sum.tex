%!TEX root = ../main.tex

\chapter{总结和展望}

\section{工作总结}
随着计算机图形学、计算机辅助设计和多媒体技术的发展,对三维模型的精度要求越来越高,三维模型数据量越来越大,因此,网格简化特别是精度可控的简化成为一个热门的研究课题。普通的简化方法虽然能够满足一般的应用需求,但是,很难做到精度可控。本文主要介绍了一种新型的鲁棒的网格同坯近似简化方法以及我们对该方法的改进。该方法主要可以分为如下步骤:
\begin{enumerate}[(1)]
  \item 细化,构建误差空间Ω的近似Γ;
  \item 简化误差空间Γ的边界;
  \item 镶嵌 0-等值点;
  \item 简化 0-等值面;
  \item 所有可能的简化。
\end{enumerate}
在细化阶段,以包围球的顶点和内外壳的采样点构建四面体网格结构$\tau$,并在其上维护一个误差函数$\varepsilon(s)$,通过贪心地加入误差最大的采样点,直到所有内外壳上的采样点均满足误差约束条件;简化误差空间$\Gamma$的边界即通过在保证误差条件的情况下对∂Γ进行消边来简化网格;然后镶嵌0-等值点,并通过0-等值面上的消边来简化网格;最后再通过对所有边进行消边来简化网格。与其他基于内外壳的网格简化算法不同,该算法以给定误差空间的内外边界上的采样点为输入,通过空间四面体化,并通过这个四面体网格中线性差值维护一个空间中的误差函数$\varepsilon(s)$,保证简化结果能够将内外边界的采样点分开,从而控制简化结果与原模型之间的误差在用户给定的范围内。然而,该方法在第一阶段细化时并没有去考虑原网格上的各项异性信息(使用3D Delaunay三角化来生成四面体网格),因此需要很多后序的消边操作。我们改进了初始化算法,使用考虑了各项异性的三角化方法——对原网格做一个使其各项同性形变,在形变的空间中做3D Delaunay三角化,再映射到原空间,得到一个带有各项异性的三角化结果。实验证明,我们的算法而优化了细化结果,不仅减少了后序的简化操作,而且优化了最终简化结果。
\section{展望}
首先,如我们的结果展示中所提到的一个问题:我们的算法可能会消耗更多的时间——由于增大了边的Kernel Region,一种可能的解决方法是通过某种减枝的方法将大部分不满足条件的候选合并点首先剔除,从而加速判断过程。另外,我们的算法的一个局限是——在我们的形变时,一旦内外壳相交或者发生自交,那么后序的算法得到的结果将无法保证,而对于一些复杂的模型我们很难控制这些相交的情况。在我们的实验结果中,我们通过手动调节参数的方式来得到一个内外壳不相交也不自交的网格。虽然我们的改进减少了后序的简化操作,该算法也可以在单个计算机上并行的计算,但是该算法还是非常地耗时,将来我们可以将一个网格进行划分,使用多个计算机上并行地计算。为了保证误差条件的一致,仍然返回到了原空间进行误差判断,而且细化之后其余操作均在原空间中进行(在选取合并点的时候在空间上的采样并没有很好地利用各项异性,而是均匀地采样,理想情况下各个方向采样间隙不一样),因此,在将来我们可以去探索一种能够保持这个精度条件一致的计算的内外壳的位置的方法,从而我们在形变的空间中做完简化,然后再映射到原空间,从而进一步提升效率。
