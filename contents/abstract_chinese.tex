% !TEX root = ../main.tex

% 定义中文摘要和关键字
\begin{cabstract}
随着工业精度要求的不断提高,三维扫描等技术的不断发展,虽然计算机的硬件性能也越来越高,但日趋庞大的三维模型数据,仍使得我们对三维模型的保存、传输、修改、以及大量三维模型的算法带来了很多不便。因此,如何在可控的精度范围内,去找到一个尽可能简化的三维模型去近似原模型是一个非常基础而重要的工作。传统的网格简化方法主要根据一个误差能量迭代的消除代价最小的边或者顶点。这种方法简单高效,但是很难提供很好的精度控制(主要靠单向的Hausdorff距离来做精度控制)。 而[Manish Mandad,David Cohen-Steiner 2015]通过维护一个误差空间的参数化函数,保证在允许的误差内外边界采样点的参数化误差不会超过1,即0等值面(最终结果)不会和允许的误差内外边界相交,从而提供了双向的hausdorff距离的精度控制。
本文基于[Manish Mandad,David Cohen-Steiner 2015]的算法并做了改进,通过将基于各项异性的细化,优化了该算法初始化状态,从而提高了该算法的效率并优化了结果。该算法大致可以分为以下几个步骤:(1)基于各向异性的细化;(2)简化误差空间Γ的边界;(2)镶嵌0等值点;(3)简化0等值面;(4)所有可能的简化。相对于原算法,我们在初始化时就根据原网格上存在的各项异性信息(在网格的一个点上的不同方向上相同的拉伸导致的误差不同)构建初始化网格,不仅使得我们的初始化状态更加精简,减少了后序的网格简化步骤,而且更好地降低与原网格之间的误差得到更优的结果。
实验结果表明我们的算法相对原算法减少了计算量提升了效率,并优化了简化结果。
%% 请注意,以下内容主要参考自薛瑞尼的清华大学论文模板。

%% 论文的摘要是对论文研究内容和成果的高度概括。
%% 摘要应对论文所研究的问题及其研究\textbf{目的}进行描述,
%% 对研究\textbf{方法和过程}进行简单介绍,
%% 对研究\textbf{成果和所得结论}进行概括。
%% 摘要应具有独立性和自明性,
%% 其内容应包含与论文全文等量的主要信息。
%% 使读者即使不阅读全文,
%% 通过摘要就能了解论文的总体内容和主要成果。

%% 论文摘要的书写应力求精确、简明。切忌写成对论文书写内容进行提要的形式,尤其要避
%% 免“第 1 章……;第 2 章……;……”这种或类似的陈述方式。
%% 不宜使用公式、图表,不标注引用文献。
%% 硕士论文摘要的字数一般为300--500 个左右。

%% 关键词是为了文献标引工作、
%% 用以表示全文主要内容信息的单词或术语。
%% 关键词不超过 5个,
%% 每个关键词中间用分号分隔
%% (研究生院要求使用分号分隔,软件学院要求使用逗号分隔)。
%% 见源码\texttt{zjuthesis.cls}搜索keywords了解。
\end{cabstract}

\ckeywords{网格简化,网格变形,各项同性,各项异性}
