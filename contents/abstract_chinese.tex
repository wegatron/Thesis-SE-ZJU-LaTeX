% !TEX root = ../main.tex

% 定义中文摘要和关键字
\begin{cabstract}
随着三维扫描技术、计算机辅助设计等的不断发展,人们对三维模型精度也提出了更高的要求,三维模型的数据量也越来越大,虽然计算机的硬件性能也越来越高。但日趋庞大的三维模型数据,仍使得我们对三维模型的保存、传输、修改、以及大量三维模型的处理算法带来了很多不便。因此,如何在可控的精度范围内,去找到一个尽可能简化的三维模型去近似原模型是一个非常基础而重要的工作。传统的网格简化方法主要根据一个误差能量,迭代地消除代价最小的边或者顶点。这种方法简单高效,能够满足一些基本需求,但是很难做到的精度可控,因此满足工业制造的需求。而Manish Mandad等人提出了一种新型的算法\cite{isotopic-appro},维护一个误差空间的参数化函数,通过控制误差内外边界采样点上的参数化函数值,来使得最终简化结果能够将内外边界采样点分开,即最终结果在误差空间中且不会和误差空间的内外边界相交,从而提严格地控制简化误差。\par
本文基于Manish Mandad等人的算法\cite{isotopic-appro}并做了改进,通过采用基于各向异性的细化,优化了该算法初始化状态,从而提高了该算法的效率并优化了结果。该算法大致可以分为以下几个步骤:(1)基于各向异性的细化;(2)简化误差空间$\Gamma$的边界;(2)镶嵌0-等值点;(3)简化0-等值面;(4)所有可能的简化。相对于原算法,我们在初始化时就根据原网格上存在的各向异性信息(在网格的一个点上的不同方向上相同的拉伸导致的误差不同)构建初始化网格,不仅使得我们的初始化状态更加精简,减少了后序的网格简化步骤,而且更好地降低与原网格之间的误差得到更优的结果。\par
实验结果表明我们的算法相对原算法减少了计算量提升了效率,并优化了简化结果。
\end{cabstract}

\ckeywords{网格简化,误差空间,标架场,网格变形,各向同性,各向异性}
