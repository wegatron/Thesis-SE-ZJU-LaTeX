\chapter{结果展示}
下面我们展示我们的算法和原在一些例子上一些结果,并与原算法做一个比较,说明我们的算法的优势。我们定义误差$\delta$为该三角网格的外接球的直径的百分比。从结果中我们可以发现,我们的形变算法能够根据原网格的各项异性信息,有效地将原网格形变成为一个更趋于各项同性的网格,从而在形变之后的网格上的3D Delaunay三角化对应于原网格上则是带有其各项异性信息的三角化方法,从而优化了细化的结果,不仅减少了后序的简化操作,而且优化了简化结果。所有的实验结果均在一台CPU为Interlxxx,内存为XXX的机器上运行得到。从这些结果中,我们可以看到,我们的算法的结果更充分利用了原模型上的各项异性信息(有更多细长的三角面片),从而能够在相同的误差范围内得到更简化的网格。

\section{Cylinder模型}
在一个细长的圆柱体模型(顶点数量为10000)上,我们的算法和原算法的结果对比:
\begin{figure}[htbp]
  \centering
  \begin{subfigure}[b]{0.4\textwidth}
    \includegraphics[width=\textwidth]{cylinder.png}
    \end{subfigure}
    \begin{subfigure}[b]{0.4\textwidth}
      \includegraphics[width=\textwidth]{cylinder_deformed.png}
    \end{subfigure}
    \caption[Cylinder形变结果]{原网格(a)和用我们的形变算法形变之后的网格(b)}
    \label{fig:cylinder-deform}
\end{figure}


\begin{figure}[htbp]
  \centering
  \begin{subfigure}[b]{0.4\textwidth}
    \includegraphics[width=\textwidth]{CYLINDER_0_0.02_0.02_0.04_100_refine.png}
  \end{subfigure}
  \begin{subfigure}[b]{0.4\textwidth}
    \includegraphics[width=\textwidth]{CYLINDER_1_0.02_0.02_0.04_100_refine.png}
  \end{subfigure}
  \begin{subfigure}[b]{0.4\textwidth}
    \includegraphics[width=\textwidth]{CYLINDER_0_0.02_0.02_0.04_100.png}
  \end{subfigure}
  \begin{subfigure}[b]{0.4\textwidth}
    \includegraphics[width=\textwidth]{CYLINDER_1_0.02_0.02_0.04_100.png}
  \end{subfigure}
  \caption[当$\delta=0.58\%$时Cylinder结果对比]{当$\delta=0.58\%$时,原算法的细化结果(左上图)和我们的算法的细化结果(右上图),原算法的简化结果(左下图,最终顶点数量为192)和我们的算法的简化结果(右下图,最终顶点数量为132)}
  \label{fig:cylinder-res1}
\end{figure}

\begin{figure}[htbp]
  \centering
  \begin{subfigure}[b]{0.4\textwidth}
    \includegraphics[width=\textwidth]{CYLINDER_0_0.01_0.005_0.02_100_refine.png}
  \end{subfigure}
  \begin{subfigure}[b]{0.4\textwidth}
    \includegraphics[width=\textwidth]{CYLINDER_1_0.01_0.005_0.02_100_refine.png}
  \end{subfigure}
  \begin{subfigure}[b]{0.4\textwidth}
    \includegraphics[width=\textwidth]{CYLINDER_0_0.01_0.005_0.02_100.png}
  \end{subfigure}
  \begin{subfigure}[b]{0.4\textwidth}
    \includegraphics[width=\textwidth]{CYLINDER_1_0.01_0.005_0.02_100.png}
  \end{subfigure}
  \caption[当$\delta=0.29\%$时Cylinder结果对比]{当$\delta=0.29\%$时,原算法的细化结果(左上图)和我们的算法的细化结果(右上图),原算法的简化结果(左下图,最终顶点数量为350)和我们的算法的简化结果(右下图,最终顶点数量为254)}
  \label{fig:cylinder-res2}
\end{figure}

%% 原网格 - 形变之后的网格
%% \delta=? 时的对比
%% \delta=? 时的对比
%% \delta=? 时时间和顶点数量列表和原来算法的对比

\section{Cigar模型}
在一个细长的雪茄体模型(顶点数量为5000)上,我们的算法和原算法的结果对比:
\begin{figure}[htbp]
  \centering
  \begin{subfigure}[b]{0.4\textwidth}
    \includegraphics[width=\textwidth]{cigar.png}
    \end{subfigure}
    \begin{subfigure}[b]{0.4\textwidth}
      \includegraphics[width=\textwidth]{cigar_deformed.png}
    \end{subfigure}
    \caption[cigar形变结果]{原网格(a)和用我们的形变算法形变之后的网格(b)}
    \label{fig:cigar-deform}
\end{figure}


\begin{figure}[htbp]
  \centering
  \begin{subfigure}[b]{0.4\textwidth}
    \includegraphics[width=\textwidth]{CIGAR_0_0.1_0.05_0.1_100_refine.png}
  \end{subfigure}
  \begin{subfigure}[b]{0.4\textwidth}
    \includegraphics[width=\textwidth]{CIGAR_1_0.1_0.05_0.1_100_refine.png}
  \end{subfigure}
  \begin{subfigure}[b]{0.4\textwidth}
    \includegraphics[width=\textwidth]{CIGAR_0_0.1_0.05_0.1_100.png}
  \end{subfigure}
  \begin{subfigure}[b]{0.4\textwidth}
    \includegraphics[width=\textwidth]{CIGAR_1_0.1_0.05_0.1_100.png}
  \end{subfigure}
  \caption[当$\delta=1.06\%$时cigar结果对比]{当$\delta=1.06\%$时,原算法的细化结果(左上图)和我们的算法的细化结果(右上图),原算法的简化结果(左下图,最终顶点数量为89)和我们的算法的简化结果(右下图,最终顶点数量为69)}
  \label{fig:cigar-res1}
\end{figure}

\begin{figure}[htbp]
  \centering
  \begin{subfigure}[b]{0.4\textwidth}
    \includegraphics[width=\textwidth]{CIGAR_0_0.03_0.015_0.03_100_refine.png}
  \end{subfigure}
  \begin{subfigure}[b]{0.4\textwidth}
    \includegraphics[width=\textwidth]{CIGAR_1_0.03_0.015_0.03_100_refine.png}
  \end{subfigure}
  \begin{subfigure}[b]{0.4\textwidth}
    \includegraphics[width=\textwidth]{CIGAR_0_0.03_0.015_0.03_100.png}
  \end{subfigure}
  \begin{subfigure}[b]{0.4\textwidth}
    \includegraphics[width=\textwidth]{CIGAR_1_0.03_0.015_0.03_100.png}
  \end{subfigure}
  \caption[当$\delta=0.32\%$时cigar结果对比]{当$\delta=0.32\%$时,原算法的细化结果(左上图)和我们的算法的细化结果(右上图),原算法的简化结果(左下图,最终顶点数量为296)和我们的算法的简化结果(右下图,最终顶点数量为268)}
  \label{fig:cigar-res2}
\end{figure}

\section{Banana模型}
在一个香蕉模型(顶点数量为12000)上,我们的算法和原算法的结果对比:
\begin{figure}[htbp]
  \centering
  \begin{subfigure}[b]{0.4\textwidth}
    \includegraphics[width=\textwidth]{banana.png}
    \end{subfigure}
    \begin{subfigure}[b]{0.4\textwidth}
      \includegraphics[width=\textwidth]{banana_deformed.png}
    \end{subfigure}
    \caption[banana形变结果]{原网格(a)和用我们的形变算法形变之后的网格(b)}
    \label{fig:banana-deform}
\end{figure}

\begin{figure}[htbp]
  \centering
  \begin{subfigure}[b]{0.4\textwidth}
    \includegraphics[width=\textwidth]{BANANA_0_0.06_0.03_0.12_100_refine.png}
  \end{subfigure}
  \begin{subfigure}[b]{0.4\textwidth}
    \includegraphics[width=\textwidth]{BANANA_1_0.06_0.03_0.12_100_refine.png}
  \end{subfigure}
  \begin{subfigure}[b]{0.4\textwidth}
    \includegraphics[width=\textwidth]{BANANA_0_0.06_0.03_0.12_100.png}
  \end{subfigure}
  \begin{subfigure}[b]{0.4\textwidth}
    \includegraphics[width=\textwidth]{BANANA_1_0.06_0.03_0.12_100.png}
  \end{subfigure}
  \caption[当$\delta=0.5\%$时banana结果对比]{当$\delta=0.5\%$时,原算法的细化结果(左上图)和我们的算法的细化结果(右上图),原算法的简化结果(左下图,最终顶点数量为248)和我们的算法的简化结果(右下图,最终顶点数量为211)}
  \label{fig:banana-res1}
\end{figure}

\begin{figure}[htbp]
  \centering
  \begin{subfigure}[b]{0.4\textwidth}
    \includegraphics[width=\textwidth]{BANANA_0_0.03_0.015_0.06_100_refine.png}
  \end{subfigure}
  \begin{subfigure}[b]{0.4\textwidth}
    \includegraphics[width=\textwidth]{BANANA_1_0.03_0.015_0.06_100_refine.png}
  \end{subfigure}
  \begin{subfigure}[b]{0.4\textwidth}
    \includegraphics[width=\textwidth]{BANANA_0_0.03_0.015_0.06_100.png}
  \end{subfigure}
  \begin{subfigure}[b]{0.4\textwidth}
    \includegraphics[width=\textwidth]{BANANA_1_0.03_0.015_0.06_100.png}
  \end{subfigure}
  \caption[当$\delta=0.25\%$时banana结果对比]{当$\delta=0.25\%$时,原算法的细化结果(左上图)和我们的算法的细化结果(右上图),原算法的简化结果(左下图,最终顶点数量为474)和我们的算法的简化结果(右下图,最终顶点数量为437)}
  \label{fig:banana-res2}
\end{figure}

\section{大白模型}
在大白模型(顶点数量为20000)上,我们的算法和原算法的结果对比:
\begin{figure}[htbp]
  \centering
  \begin{subfigure}[b]{0.4\textwidth}
    \includegraphics[width=\textwidth]{creature.png}
    \end{subfigure}
    \begin{subfigure}[b]{0.4\textwidth}
      \includegraphics[width=\textwidth]{creature_deformed.png}
    \end{subfigure}
    \caption[大白的形变结果]{原网格(a)和用我们的形变算法形变之后的网格(b)}
    \label{fig:creature-deform}
\end{figure}

\begin{figure}[htbp]
  \centering
  \begin{subfigure}[b]{0.4\textwidth}
    \includegraphics[width=\textwidth]{CREATURE_0_0.05_0.025_0.1_100_refine.png}
  \end{subfigure}
  \begin{subfigure}[b]{0.4\textwidth}
    \includegraphics[width=\textwidth]{CREATURE_1_0.05_0.025_0.1_100_refine.png}
  \end{subfigure}
  \begin{subfigure}[b]{0.4\textwidth}
    \includegraphics[width=\textwidth]{CREATURE_0_0.05_0.025_0.1_100.png}
  \end{subfigure}
  \begin{subfigure}[b]{0.4\textwidth}
    \includegraphics[width=\textwidth]{CREATURE_1_0.05_0.025_0.1_100.png}
  \end{subfigure}
  \caption[当$\delta=0.22\%$时banana结果对比]{当$\delta=0.22\%$时,原算法的细化结果(左上图)和我们的算法的细化结果(右上图),原算法的简化结果(左下图,最终顶点数量为958)和我们的算法的简化结果(右下图,最终顶点数量为852)}
  \label{fig:creature-res1}
\end{figure}

\begin{figure}[htbp]
  \centering
  \begin{subfigure}[b]{0.4\textwidth}
    \includegraphics[width=\textwidth]{CREATURE_0_0.025_0.015_0.05_100_refine.png}
  \end{subfigure}
  \begin{subfigure}[b]{0.4\textwidth}
    \includegraphics[width=\textwidth]{CREATURE_1_0.025_0.015_0.05_100_refine.png}
  \end{subfigure}
  \begin{subfigure}[b]{0.4\textwidth}
    \includegraphics[width=\textwidth]{CREATURE_0_0.025_0.015_0.05_100.png}
  \end{subfigure}
  \begin{subfigure}[b]{0.4\textwidth}
    \includegraphics[width=\textwidth]{CREATURE_1_0.025_0.015_0.05_100.png}
  \end{subfigure}
  \caption[当$\delta=0.11\%$时banana结果对比]{当$\delta=0.11\%$时,原算法的细化结果(左上图)和我们的算法的细化结果(右上图),原算法的简化结果(左下图,最终顶点数量为1911)和我们的算法的简化结果(右下图,最终顶点数量为1725)}
  \label{fig:creature-res2}
\end{figure}

\section{Fertility模型}
在Fertility模型(顶点数量为13971)上与原算法的结果对比:
\begin{figure}[htbp]
  \centering
  \begin{subfigure}[b]{0.4\textwidth}
    \includegraphics[width=\textwidth]{fertility.png}
    \end{subfigure}
    \begin{subfigure}[b]{0.4\textwidth}
      \includegraphics[width=\textwidth]{fertility_deformed.png}
    \end{subfigure}
    \caption[fertility形变结果]{原网格(a)和用我们的形变算法形变之后的网格(b)}
    \label{fig:fertility-deform}
\end{figure}

\begin{figure}[htbp]
  \centering
  \begin{subfigure}[b]{0.4\textwidth}
    \includegraphics[width=\textwidth]{FERTILITY_0_0.5_0.25_1.5_100_refine.png}
  \end{subfigure}
  \begin{subfigure}[b]{0.4\textwidth}
    \includegraphics[width=\textwidth]{FERTILITY_1_0.5_0.25_1.5_100_refine.png}
  \end{subfigure}
  \begin{subfigure}[b]{0.4\textwidth}
    \includegraphics[width=\textwidth]{FERTILITY_0_0.5_0.25_1.5_100.png}
  \end{subfigure}
  \begin{subfigure}[b]{0.4\textwidth}
    \includegraphics[width=\textwidth]{FERTILITY_1_0.5_0.25_1.5_100.png}
  \end{subfigure}
  \caption[当$\delta=0.1948\%$时fertility结果对比]{当$\delta=0.1948\%$时,原算法的细化结果(左上图)和我们的算法的细化结果(右上图),原算法的简化结果(左下图,最终顶点数量为1784)和我们的算法的简化结果(右下图,最终顶点数量为1640)}
  \label{fig:fertility-res1}
\end{figure}

\begin{figure}[htbp]
  \centering
  \begin{subfigure}[b]{0.4\textwidth}
    \includegraphics[width=\textwidth]{FERTILITY_0_0.25_0.125_1_100_refine.png}
  \end{subfigure}
  \begin{subfigure}[b]{0.4\textwidth}
    \includegraphics[width=\textwidth]{FERTILITY_1_0.25_0.125_1_100_refine.png}
  \end{subfigure}
  \begin{subfigure}[b]{0.4\textwidth}
    \includegraphics[width=\textwidth]{FERTILITY_0_0.25_0.125_1_100.png}
  \end{subfigure}
  \begin{subfigure}[b]{0.4\textwidth}
    \includegraphics[width=\textwidth]{FERTILITY_1_0.25_0.125_1_100.png}
  \end{subfigure}
  \caption[当$\delta=0.0974\%$时fertility结果对比]{当$\delta=0.0974\%$时,原算法的细化结果(左上图)和我们的算法的细化结果(右上图),原算法的简化结果(左下图,最终顶点数量为3832)和我们的算法的简化结果(右下图,最终顶点数量为3550)}
  \label{fig:fertility-res2}
\end{figure}

\section{对比信息综合}
\autoref{tab:compare_res}是这些模型上的对比信息的综合,从表格中我们可以看出我们的算法所消耗的时间可能会多余原算法,原因是我们通过各项异性的细化所得到的初始化状态,增大了各条边的Kernel Region,从而在整个简化过程中们需要更多次地判断候选合并点是否满足误差约束条件,以及候选合并点是否满足法向约束条件,而这两个判断我们需要求解多个线性方程,因此也恰恰是最耗时的。所以我们的算法在优化了简化结果的同时,可能会增加算法运行的时间。
\begin{table}[htbp]
     \centering
     \caption{我们的算法和在各种模型上与原算法的对比}\label{tab:compare_res}
    \begin{tabu}{lcccc} % lrr 表示 左对齐 右对齐 右对齐
    \toprule
    \textbf{模型/$\delta$}  & \textbf{原算法顶点数} & \textbf{现算法顶点数} & \textbf{原时间}(秒) & \textbf{现时间}(秒)\\
    \midrule
    cylinder/$\delta=0.58\%$     & 192  & 132 & 240 & 261 \\
    cylinder/$\delta=0.29\%$     & 350  & 254 & 4578 & 8363\\
    cigar/$\delta=1.06\%$        & 89 & 69 & 243 & 287 \\
    cigar/$\delta=0.32\%$        & 296 & 268 & 9002 & 11636  \\
    banaba/$\delta=0.5\%$        & 248 & 211 & 430 & 579 \\
    banana/$\delta=0.25\%$       & 474 & 437 & 8637 & 8599 \\
    creature/$\delta=0.22\%$     & 958 & 852 & 6881 & 6013 \\
    creature/$\delta=0.11\%$     & 1911 & 1725 & 119007 & 78510 \\
    fertility/$\delta=0.1948\%$     & 1784 & 1640 & 8314 & 4673 \\
    fertility/$\delta=0.1948\%$     & 3832 & 3550 & 48515 & 53027 \\
    \bottomrule
    \end{tabu}%
\end{table}
