% !TEX root = ../main.tex

% 定义英文摘要和关键字

\begin{eabstract}
With the continuous development of the 3D scanning technology and computer-aided design, people also put forward higher requirements on the precision of 3D model.
Though the hardware is improved and improving, but the increasingly large 3D model data is still difficult for us to save, transfer and edit. Also it makes mesh processing a time-consuming and difficult task.
So to find a simplified 3d model to approximate the original model in a controllable tolerance is a basic and important work.
Traditional mesh simplification method is mainly to iteratively eliminate the minimum cost edge or vertex according to an error energy. These method is simple and efficient, are able to meet some basic needs, but are unable to provide reliable precision control to meet the needs of industrial manufacturing.
Manish Mandad et al proposed a new algorithm \cite{isotopic-appro}, by maintaining a parameterized function in tolerance volume and controling the parameterized function value on each sample points in the tolerance boundary, making the final simplified mesh will classsify all the samples on the inner and outer boundary as good, they make the final simplified mesh in the specified tolerance volume.\par

We improve Manish Mandad et al's algorithm by using anisotropic refinement which will optimize the algorithm's initial state, to improve the efficiency of the algorithm and optimize the result. Our algorithm can be roughly divided into the following steps: (1) Anisotropic refinement; (2) Simplify Tolerance; (2) Mutual-tessellation; (3) Simplify zero-surface; (4) All edges collapse. By using the anisotropic information (at one point of the triangle mesh, a same stretch in different directions may lead to different errors) to leed the construction of the initial tetrahedral mesh in refine step, not only simplified the initial state, reducing the subsequent simplifications, but also improve the final result —— makes the final result more simplified.\par
The experimental result show that our algorithm reduces the calculation, increases efficiency, and improve the results.
\end{eabstract}

\ekeywords{Mesh simplification, tolerance volume, frame field, mesh warping, isotropic, anisotropic}
