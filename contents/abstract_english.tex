% !TEX root = ../main.tex

% 定义英文摘要和关键字

\begin{eabstract}
With the continuous development of the 3D scanning technology and computer-aided design, people also put forward higher requirements on the precision of 3D model. Though the computer hardware are being improving, but the increasingly large 3D model data still difficult for us to save, transfer. Also it makes mesh processing a time-consuming and difficult task.
So to find a simplified 3d model to approximate the original model in the controllable accuracy range is a very basic and important work. Traditional mesh simplification method is mainly to eliminate the minimum cost edge or vertices according to an error energy. This method is simple and efficient, able to meet some basic needs, but it is unable to provide strict precision control to meet the needs of industrial manufacturing. Manish Mandad et al proposed a new algorithm \cite{isotopic-appro}, by maintaining a parameterized function in tolerance volume and control the value of the parameterized function value on each sample points in the tolerance boundary, to make the final simplified mesh will classsify inner and outer tolerance boundary as good. In this way the final simplified mesh will in tolerance volume.\par

Based on Manish Mandad et al's algorithm \cite{isotopic-appro} and improved through using anisotropic refinement which optimized the algorithm initialization state, thereby improving the efficiency of the algorithm and optimization of the results. The algorithm can be roughly divided into the following steps: (1) Based on Anisotropic refinement; (2) to simplify the space $\Gamma$ boundary error; (2) mutual-tessellation; (3) simplify the zero-surface; (4) all edges collapse. By using the anisotropic information(in different directions at one point of the grid on the same stretch lead to different error)to leed the construction of the initial tetmesh in refine step, not only simplified the initial state, reducing the subsequent simplifications, but also improve the final result —— makes the final result more simplified.\par
The experimental result show that our algorithm reduces the calculation, increases efficiency, and improve the results.
\end{eabstract}

\ekeywords{Mesh simplification, tolerance volume, frame field, mesh warping, isotropic, anisotropic}
